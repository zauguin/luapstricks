%% $Id: pst-tools.tex 245 2021-09-13 18:56:41Z herbert $
%%
%% This is file `pst-tools.tex',
%%
%% IMPORTANT NOTICE:
%%
%% Package `pst-tools.tex'
%%
%% Herbert Voss <hvoss@tug.org>
%%
%% This program can be redistributed and/or modified under the terms
%% of the LaTeX Project Public License Distributed from CTAN archives
%% in directory macros/latex/base/lppl.txt.
%%
%% DESCRIPTION:
%%   `pst-tools' is a PSTricks package for helper functions
%%
%%
\csname PSTtoolsLoaded\endcsname
\let\PSTtoolsLoaded\endinput

\ifx\PSTricksLoaded\endinput\else\input pstricks.tex\fi
\ifx\PSTXKeyLoaded\endinput\else \input pst-xkey.tex \fi
%
\edef\PstAtCode{\the\catcode`\@} \catcode`\@=11\relax
% interface to the `xkeyval' package
\pst@addfams{pst-tools}
%
\def\fileversion{0.11}
\def\filedate{2021/09/13}
\message{`PST-tools' v\fileversion, \filedate\space (hv)}
%
\define@key[psset]{pst-tools}{decimalSeparator}[.]{\def\psk@decimalSeparator{#1}}%
\define@boolkey[psset]{pst-tools}[Pst@]{comma}[true]{%
  \ifPst@comma\def\psk@decimalSeparator{,}\else\def\psk@decimalSeparator{.}\fi}
\define@boolkey[psset]{pst-tools}[Pst@]{trimSpaces}[true]{}
\define@key[psset]{pst-tools}{xShift}[0]{\def\psk@xShift{#1}}
\define@key[psset]{pst-tools}{yShift}[0]{\def\psk@yShift{#1}}
\define@key[psset]{pst-tools}{postString}[{}]{\def\psk@postString{#1}}
\define@key[psset]{pst-tools}{VarName}[]{\def\psk@VarName{#1 }}
\define@key[psset]{pst-tools}{PSfont}[NimbusRomNo9L-Regu]{%
  \ifPSTlualatex
    \ifx\relax#1\relax
       \def\psk@printfont{ \fontid\font\space .findfontid }%
    \else
      \def\psk@printfont{ /#1 findfont }% 
     \fi
  \else
    \ifx\relax#1\relax
       \def\psk@printfont{ /NimbusRomNo9L-Regu  findfont }%
    \else
       \def\psk@printfont{ /#1 findfont }% 
    \fi
  \fi
}%
\define@key[psset]{pst-tools}{valuewidth}[10]{\pst@getint{#1}\psk@valuewidth }
\define@key[psset]{pst-tools}{fontscale}[10]{\pst@checknum{#1}\psk@fontscale }
\define@key[psset]{pst-tools}{decimals}[-1]{\pst@getint{#1}\psk@decimals }
\psset[pst-tools]{PSfont=Helvetica,fontscale=10,valuewidth=10,decimals=-1,
                  xShift=0,yShift=0,trimSpaces=false,postString={},VarName={},
                  comma=false,decimalSeparator=.}
%
\def\txG{ true setglobal globaldict begin }
\def\etxG{ end false setglobal }
%
\newpsstyle{mmpaper}{subgriddiv=5,gridlabels=0,gridwidth=1pt,gridcolor=orange,subgridwidth=0.1pt,subgridcolor=orange!90}
%
\def\psPrintValue{\pst@object{psPrintValue}}
\def\psPrintValue@i#1{\expandafter\psPrintValue@ii#1,,\@nil}
\def\psPrintValue@ii#1,#2,#3\@nil{%  #1,#2 only for algebraic code
  \begin@SpecialObj
  \addto@pscode{  
     gsave \psk@printfont 
       \psk@fontscale scalefont setfont 
     \ifPst@algebraic 
       /x #1 def 
       /Func (#2) tx@AlgToPs begin AlgToPs end cvx def 
       Func 
     \else 
       #1 
     \fi
     \ifx\relax\psk@VarName\relax\else dup \txG 10 string cvs /\psk@VarName\space exch def \etxG \fi
     \psk@decimals -1 gt { 10 \psk@decimals exp dup 3 1 roll mul cvi exch div } if
     \psk@valuewidth string cvs %/Output exch def % save output
     %/concatstringarray  %  [(a) (b) ... (z)] --> (ab...z)  20100422
     \ifx\relax\psk@postString\relax\else [ exch (\psk@postString) ] concatstringarray \fi
     \ifPst@trimSpaces rightTrim \fi
     \ifPst@comma dot2comma \fi        % do we have to change dot to comma
     \psk@xShift\space \psk@yShift\space moveto  %Output 
     show grestore
  }%
  \end@SpecialObj
}

\define@boolkey[psset]{pst-tools}[Pst@]{round}[true]{}%
\define@boolkey[psset]{pst-tools}[Pst@]{science}[true]{%
  \ifPst@science\def\psk@Scin{true }\else\def\psk@Scin{false }\fi}
\psset[pst-tools]{science=false,round=false}
%
\def\psPrintValueNew{\pst@object{psPrintValueNew}}
\def\psPrintValueNew@i#1{\expandafter\psPrintValueNew@ii#1,,!!}
\def\psPrintValueNew@ii#1,#2,#3!!{%  #1,#2 only for algebraic code
  \begin@SpecialObj
  \addto@pscode{  %		thanks to Buddy Ledger
     /mfont { \psk@PSfont 
              \psk@fontscale scalefont setfont } bind def
     /mfontexp { \psk@PSfont
                 \psk@fontscale 1.2 div scalefont setfont } bind def
     /s1 { /StandardSymL findfont \psk@fontscale scalefont setfont } bind def
     \ifPst@algebraic 
        /x #1 def
        /Func (#2) tx@AlgToPs begin AlgToPs end cvx def 
        Func  
     \else #1 \fi
     /value exch def
     \psk@Scin {
       value 0 ne { value log floor cvi /expon exch def }{ /expon 0 def } ifelse
       value 10 expon exp div 
       \psk@decimals -1 gt { 10  \psk@decimals exp dup 3 1 roll mul 
         \ifPst@round round \else cvi \fi  exch div } if
       \psk@decimals 0 eq { cvi } if /numb exch def
       expon \psk@valuewidth string cvs /expon exch def
       numb \psk@valuewidth string cvs 
       \ifPst@comma dot2comma \fi  % do we have to change dot to comma
       /Output exch def
       /txspc \psk@fontscale 4 div def
       \psk@xShift\space 0 moveto mfont Output show
       txspc 0 rmoveto s1 (\string\264) show
       txspc 0 rmoveto mfont (10) show
       txspc 2 div txspc 1.5 mul rmoveto mfontexp expon show }
     { value
       \psk@decimals -1 gt { 10 \psk@decimals exp dup 3 1 roll mul 
         \ifPst@round round \else cvi \fi exch div } if
       \psk@decimals 0 eq { cvi } if %inserted to handle decimals=0
       \psk@valuewidth string cvs 
       \ifPst@comma dot2comma \fi         % do we have to change dot to comma
       \psk@xShift\space 0 moveto mfont %Output 
       show
     } ifelse
  }%
  \end@SpecialObj
}
%
\def\psRegisterList#1#2{%
  \@namedef{#1}##1{\@nameuse{#1##1}}%
  \pst@cnto=1
  \@for\pst@tempA:=#2\do{%
    \edef\pst@tempB{\@namedef{#1\the\pst@cnto}{\pst@tempA}}%	
    \pst@tempB
    \advance\pst@cnto by 1
  }%
}
% RANDOM.TEX       v.1   (Donald Arseneau)
% Generating "random" numbers in TeX. 
%
% Random integers are generated in the range 1 to 2147483646 by the
% macro \nextrandom.  The result is returned in the counter \randomi.
% Do not change \randomi except, perhaps, to initialize it at some
% random value.  If you do not initialize it, it will be initialized
% using the time and date.  (This is a sparse initialization, giving
% fewer than a million different starting values, but you should use
% other sources of numbers if they are available--just remember that
% most of the numbers available to TeX are not at all random.)
%
% The \nextrandom command is not very useful by itself, unless you
% have exactly 2147483646 things to choose from.  Much more useful
% is the \setrannum command which sets a given counter to a random
% value within a specified range.  There are three parameters:
% \setrannum {<counter>} {<minimum>} {<maximum>}.  For example, to
% simulate a die-roll: \setrannum{\die}{1}{6} \ifcase\die... .
%
% If you need random numbers that are not integers, you will have to
% use dimen registers and \setrandimen.  For example, to set a random
% page width: \setrandimen \hsize{3in}{6.5in}.  The "\pointless" macro
% will remove the "pt" that TeX gives so you can use the dimensions
% as pure `real' numbers.  In that case, specify the range in pt units.
% For example,
%   \setrandimen\answer{2.71828pt}{3.14159pt}
%   The answer is \pointless\answer.
%
% The random number generator is the one by Lewis, Goodman, and Miller
% (1969) and used as "ran0" in "Numerical Recipies" using Schrage's
% method for avoiding overflows.  The multiplier is 16807 (7^5), the
% added constant is 0, and the modulus is 2147483647 (2^{31}-1).  The
% range of integers generated is 1 - 2147483646.  A smaller range would
% reduce the complexity of the macros a bit, but not much--most of the
% code deals with initialization and type-conversion.  On the other hand,
% the large range may be wasted due to the sparse seed initialization.

\newcount\randomi % the random number seed (while executing)
\global\randomi\catcode`\@  % scratch variable during definitions
\catcode`\@=11

\def\nextrandom{\begingroup
 \ifnum\randomi<\@ne % then initialize with time
    \global\randomi\time
    \global\multiply\randomi388 \global\advance\randomi\year
    \global\multiply\randomi31 \global\advance\randomi\day
    \global\multiply\randomi97 \global\advance\randomi\month
    \message{Randomizer initialized to \the\randomi.}%
    \nextrandom \nextrandom \nextrandom
 \fi
 \count@ii\randomi
 \divide\count@ii 127773 % modulus = multiplier * 127773 + 2836
 \count@\count@ii
 \multiply\count@ii 127773
 \global\advance\randomi-\count@ii % random mod 127773
 \global\multiply\randomi 16807
 \multiply\count@ 2836
 \global\advance\randomi-\count@
 \ifnum\randomi<\z@ \global\advance\randomi 2147483647\relax\fi
 \endgroup
}

\countdef\count@ii=2 % use only in boxes!
\ifx\@tempcnta\undefined \csname newcount\endcsname \@tempcnta \fi
\ifx\@tempcntb\undefined \csname newcount\endcsname \@tempcntb \fi

\def\setrannum#1#2#3{% count register, minimum, maximum
 \@tempcnta#3\advance\@tempcnta-#2\advance\@tempcnta\@ne
 \@tempcntb 2147483645 %  =  m - 2  =  2^{31} - 3
 \divide\@tempcntb\@tempcnta
 \getr@nval
 \advance\ranval#2\relax
 #1\ranval
}

\def\setrandim#1#2#3{% dimen register, minimum length, maximum length
 \dimen@#2\dimen@ii#3\relax
 \setrannum\ranval\dimen@\dimen@ii
 #1\ranval sp\relax
}

\def\getr@nval{% The values in \@tempcnta and \@tempcntb are parameters
 \nextrandom
 \ranval\randomi \advance\ranval\m@ne \divide\ranval\@tempcntb
 \ifnum\ranval<\@tempcnta\else \expandafter\getr@nval \fi
}

\def\pointless{\expandafter\PoinTless\the}
{\catcode`p=12 \catcode`t=12 
\gdef\PoinTless#1pt{#1}}

\catcode`\@=\randomi
\global\randomi=0
\newcount\ranval

\catcode`\@=\PstAtCode\relax
%
%% END: pst-tools.tex
\endinput
%
